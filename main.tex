


\documentclass[11pt,a4paper,sans]{moderncv} % Font sizes: 10, 11, or 12; paper sizes: a4paper, letterpaper, a5paper, legalpaper, executivepaper or landscape; font families: sans or roman

\moderncvstyle{banking} % CV theme - options include: 'casual' (default), 'classic', 'oldstyle' and 'banking'
\moderncvcolor{blue} % CV color - options include: 'blue' (default), 'orange', 'green', 'red', 'purple', 'grey' and 'black'

\usepackage{tgtermes}
\usepackage{setspace}
\usepackage{lettrine}
\usepackage[a4paper,left=3cm,right=2.5cm,top=2.2cm,bottom=2.8cm]{geometry}


\firstname{} % Your first name
\familyname{} % Your last name

% All information in this block is optional, comment out any lines you don't need
\title{Statement of Research Interests}
\address{Anuroop Gaddam}{}
\mobile{(+64) 21 046 8363}
\email{anuroop24@gmail.com}

%\extrainfo{additional information}
%\photo[70pt][0.4pt]{pictures/picture} % The first bracket is the picture height, the second is the thickness of the frame around the picture (0pt for no frame)
%\quote{"A witty and playful quotation" - John Smith}

%----------------------------------------------------------------------------------------

\begin{document}
\makecvtitle % Print the CV title
%----------------------------------------------------------------------------------------
%	EDUCATION SECTION
%----------------------------------------------------------------------------------------

\section{Background and Research Philosophy}
\lettrine[lraise=0.0, nindent=0em, slope=-.2em]{I}{n} the past 10 years, my career has always revolved around research and
development, starting from dabbling with sensors and micro-controllers in the early days, switching to designing wireless sensor networks, protocols, implementation and evaluation of remote wireless sensing. In particular, my research has moved beyond analysis, hardware designing and network simulations to system test-bedding in real environments. In doing so, I have come to appreciate the importance of balancing the impact of real-world environmental conditions and constraints while trying to challenge the frontiers of technology. 

{\hskip 2em}Details of my previous research work are available in my CV, but I would like to highlight my research interests can be summarized as, \par 
\par "\emph{\textbf{The use of Internet-of-Things technologies, designing ubiquitous low-cost wireless sensors networks and reliable software platforms, from which information derived can be used to enhance existing services or drive new user-centric services to benefit the vulnerable segment in our community}}.""\\
%----------------------------------------------------------------------------------------
\section{Recent Research Interest/Themes}
In the past 7 years, my research interests have centered around the design,
implementation and evaluation of wireless sensor networks. In particular, the key themes are: 
\begin{enumerate}
  \item \textbf{Wireless sensor network for wellness determination of the elderly in a smart home.} \par My research interests includes exploring  methodologies to design and develop an efficient wireless sensor network to recognize the behavior of people living alone. Using Wireless sensor and Internet of Things to determine the wellness of an elderly person living alone in their own home using a robust, flexible and data-driven Artificially Intelligent (AI) framework. To develop and design a framework integrating temporal and spatial contextual information for determining the wellness of an  venerable people living alone.   A novel behavior detection process based on the observed the data in performing essential daily activities has been designed and developed. Using an optimum number of wireless sensor units in order to reduce the complexity, cost of the whole unit without compromising operational efficiency was the focal point of the research.  The developed rule engine can update the behavior knowledge base and simultaneously execute the tasks to explore the intricacies of the generated behavior pattern. An initial decline or change in regular daily activities can suggest changes to the health and functional abilities of the elderly person.
\par {\hskip 2em}In summary,  the developed system is used to forecast the behavior and quantitative wellness of the elderly by monitoring the daily usages of household appliances using smart sensors. Wellness determination models are tested at various elderly houses, and the experimental results related to the identification of daily activities and wellness determinations are encouraging. The wellness models are updated based on the time series analysis formulations. The electronic data processing system was designed in such a way that to incorporate the Internet of Things (IoT) framework for sensing different devices, understand and act according to the requirement of the smart home environment. 

\item \textbf{IoT based Smart Therapeutic Outdoor Furniture for Gait
and Mobility Assessment of Elderly.}\par I am currently involved in designing IoT based Smart Therapeutic Outdoor Furniture with integrated wireless sensors. This system is centered to be designed to be developed to be integrated into New Zealand landscapes. A smart bench prototype has been developed and constructed to allow testing of the embedded wireless sensors that are able to capture relevant data. The readings taken by the bench will be of sit-to-stand exercise over a one-minute period, using a standard time span allows for comparative analysis of progression and regression of the ability of the user to stand up from the bench. The data by the sensors on the smart beach gives an indication of the users leg muscle and core strength. By combining the data from the smart benches and the proposed "smart pathway" (public pathway near supermarkets with embedded wireless sensors)  would help user's health care professional in recognizing balance and gait abnormalities and thereby enables early diagnosis of gait disorders.
  \item \textbf{IoT and Wireless Sensor Networks for Environmental Monitoring.} \par In this research,  a wireless sensor network based real-time drought monitoring system was developed and was successfully tested in New Zealand conditions. The developed system is a technological intervention to monitor basic information about the weather and soil condition in real-time in order to identify, predict drought conditions. Using the data collected by the remote sensors in conjunction with predictive algorithms helped the system to identify soil condition in orchards and report of any drought condition. This monitoring system helped identify drought in the early stages and prompted farmers to take corrective measures before its too late.The wireless sensor network is designed using a low-power Ultra Narrow Band RF communication device to transmit the collected data to the base station, which then uploaded the data every two seconds to the cloud for further analysis and reporting. After processing the collected data it is then displayed to the user using Google Apps to be accessed from anywhere.  
  \item \textbf{Designing e-Learning tools: Using Culturally Relevant Approach to Encourage School Children Learn Computer Science Concepts.} \par I am currently involved in the research to create tools to improve teaching and use of games and gaming methods in education, and especially for engaging native language speaking students in Computer Science learning. The research methodologies investigate how game educators can utilize to empower and encourage pupils to learn computer science concepts at schools within a M\={a}ori cultural context. The goal of this study is to motivate and establish a culturally relevant learning climate in which school children from underrepresented populations are intrinsically encouraged to engage in a discipline where they historically are underrepresented. The research aims to utilize the learner's cultural referents as a medium to develop the e-Learning tools to ensure school children gain access to study coding while maintaining their cultural integrity. 
\end{enumerate}
\section{ Future Research Direction and Research Goals}
Over the past few years, a paradigm shift has occurred in the health care industry due to disruptive technologies and sensing technologies. To address the new challenges of the era, I intend to pursue the following research directions: 
\begin{enumerate}
\item To design and development of a framework using a heterogeneous low-cost wireless sensors,  Internet of Things (IoT) and to develop AI based smart software engine to analyze the captured data to determine the ecological momentary assessment (EMA) status of a person who is experiencing ongoing mental health issues. 
\item Design and develop low-cost wireless sensor networks and protocols for remote monitoring.
\item Wearable sensors for health monitoring.
\item IoT-enabled Activities of Daily Living sensing for personalized care and intervention for elderly living alone
\item IoT-enabled holistic care provisioning to improve elderly and caregiver wellbeing
\end{enumerate}

\end{document}
